\label{chap:chapter-1}

\begin{center}
	\begin{singlespace}
		\addcontentsline{toc}{section}{Abstract} % insert into TOC
		\label{ssec:abstract}

		\MakeUppercase{\mytitleA}\\
    	\bigskip
    	\MakeUppercase{\mytitleB}\\
		\bigskip
		Abstract\\
		\bigskip \bigskip \bigskip
		by \myname, Ph.D.\\
		Washington State University\\
		\defensemonth\, \currentyear \\
		\bigskip \bigskip \bigskip
		Chair: \mychair	
	\end{singlespace}
\end{center}
  
% *****************************************************************

Neutron star-black hole binaries are one of the primary sources of 
gravitational waves.  
These systems can also produce high-powered,
bright electromagnetic counterparts via short-duration gamma
ray bursts and kilonovae, the latter of which is produced by the ejecta and is a central source of heavy r-process elements.
We consider systems where we assume the same initial black hole mass
and spin for all simulations, varying the equation of state of the neutron
star companion.
We use three finite-temperature, composition-dependent, nuclear-theory based
equations of state (SFHo, DD2, FSU2.1) and assume neutron star masses
in the range $(1.2 - 1.4)\,M_\odot$.
We show that most ejecta masses agree with predictions fit from simpler 
equations of state within the expected spread, although not all, while the ejecta velocities do agree with the 
updated fitting-formula.
We also determine that distinguishing the evolution of bound matter between two phases, early admixture and the later-time secularized classical accretion disk phase,
could be important in understanding the origins of 
gamma ray bursts.
In the earlier phase, the infusion of bound tail material (fallback) and the early-stage circularization of fluid near the horizon (protodisk) is highly energetic and bright in neutrinos \mbox{$L_{\nu} \sim (1.2 - 5.6)\times10^{53}\,{\rm erg\,s}^{-1}$}, with higher compactnesses leading to more luminous disks.
The feature-rich equations of state used in this study do have an imprint on protodisk structure, although likely only via differing compactnesses.

% *****************************************************************

\newpage