\chapter{Hot Nuclear Equations of State Simulations}
\label{chap:chapter-4}

In this chapter, we review the literature of nuclear relativity in the context of black hole-neutron star mergers using finite temperature, nuclear theory based Equations of State.  A robust general relativity code is needed not only to model the gravitational waveform, but also to produce the electromagnetic signals that can be detected by optical instruments.  In this sense, the coalescence of black hole neutron star binaries and their mergers are essential systems that can be probed with multi-messenger astronomy.  

% Foucart et al (2014)

While the inspiral of these systems can be adequately modeled with a simpler, approximate equation of state like a piecewise polytrope, the disruption phase requires a more detailed equation of state with composition and temperature dependence to more fully describe the physical effects.  To produce the gravitational signal, one must also evolve the system in the last tens of orbits prior to disruption.  On the other hand, to fully predict the electromagnetic signal and the nucleosynthesis yields, the number of requisite orbits can be relaxed.  In addition to the dependencies of neutron star structure (temperature, composition, pressure) to the nuclear reactions, we also have to accurately model magnetic fields effects of the post merger remnant disk and neutrino radiation on the disrupting fluid (and as a significant source of cooling on the disk).

Neutrinos can also play a role on the composition of the unbound material ($u_t < -1$) that gets ejected from the system, as well as the heavy r-process elements created there during disruption. Neutrino-antineutrino annihilation in the less dense regions and near the poles of the black hole can contribute to the effects on the detectable production of short gamma ray bursts.  While magnetic fields and the magneto-rotational instability (MRI) are crucial to the late-stage evolution of the accretion disk, simulations have shown that they do not play an essential role during merger.  The equation of state, however, plays a significant role in all stages of evolution: compared to a stiff equation of state, the relative response of a soft star to the tidal diruption from the black hole during inspiral occurs closer to the black hole (likewise, if at all), can lead to qualitatively different features during merger (thinner stream of matter falling into the black hole), and the multivariate dependence of the equation of state on the evolution of the remnant accretion disk are all essential.

In [cite: foucart, deaton 2014], the first simulations of finite temperature equations of state in black hole neutron mergers with an accompanying neutrino leakage scheme were performed.  This work used the Lattimer-Swesty equation of state discussed in Section \ref{sec:ls220}, and explored the parameter space where the mass of the neutron star is in the lower mass range ($M_{NS} = 1.2 M_\odot - 1.4 M_\odot$), while the black hole mass is in the peak of the distribution of likely stellar-mass black holes ($M_{BH} = 7 M_\odot - 10 M_\odot$).  For these lower-mass black holes, a larger spin is necessary to produce maximal disruption ($\chi_{BH} = 0.7 - 0.9$).  Their survey was also restricted in the sense that magnetic fields were not included since the disk wasn't evolved far out, noting that the effects of neutrino cooling on the inner regions of the disk and the viscous heating due to the MRI turbulence would more or less offset each other.

Comparing to the piecwise polytropes of Hebeler et al. [cite: 1303.4662], it was found that the use of the hot, nuclear theory based LS220 equation of state was less differentiable during merger for lower mass black holes.  For higher mass black holes, on the other hand, the difference was most notable as the more rapid disruption occurred later and closer to the black hole with a much thinner tail than with the fairly common $\Gamma = 2$ polytrope (with the same configuration parameters).

There were several restrictions noted at the time the simulations were performed.  First, the grid stricture of the finite difference grid was not adaptable at the time.  That is, given that the tail is much larger in structure than the disk, accurately resolving the accretion disk would lead to over resolving the tail--far too computationally expensive.  Conversely, lowering resolution down to accurately resolve the tail would drastically under resolve the disk.  Therefore, to evolve the disk, the large amount of unbound matter had to exit the grid unfortunately early.  Second, a full study of the disk evolution would require the effects of MRI turbulence, or an equivalent effective viscosity routine.   Lastly, the neutrino leakage scheme only goes as far as an estimate for heavy eement nucleosynthesis.  A more robust neutrino transport scheme would be needed to follow the production of neutrino driven winds and electron-positron creation in the baryon-poor polar axes of the black hole (gamma ray bursts).

Key findings from this study found that a finite temperature equation of state neutron stars reliably ejects a large amount of material ($M_\textrm{ej} = 0.04 M_\odot -  0.16 M_\odot$), where less massive stars eject more material, and form accretion disks with mass $M_\textrm{disk} = 0.05 M_\odot -  0.15 M_\odot$, where larger black hole spin lend to more massive disks (and therefore less bound tail material).  The disks are initially hot ($T_\textrm{disk} \sim 5-15 \textrm{MeV}$) and bright in neutrinos ($L_\nu \sim 10^{53} \textrm{erg/s}$).  In the first 10 ms after merger, the disk quickly protonizes from $Y_e \sim 0.06$ to $Y_e \sim 0.1 - 0.4$. 

% Deaton et al (2016)

In a follow-up, more in depth study, Deaton et al. [cite: 2016 paper] examined the effects of neutrino cooling on the remnant disk with the configuration system $M_{NS} = 1.4 M_\odot$, $M_{BH} = 5.6 M_\odot$, $\chi_{BH} = 0.9$ for the LS220 equation of state.  This was the first study of a finite-temperature, composition-dependent equation of state black hole-neutron star system with full general relativity and neutrino cooling.  Although their disk remnant was only evolved to a fraction of the accretion timescale ($\sym 200$ms), they note that the effects of MRI heating would allow the disk to remain hot long after merger, where magnetic effects can be coupled to the fluid evolution system with magnetohydrodyanmics.


In this study, the post merger remnant had baryon rest mass $M_\textrm{disk} \sim 0.3 M_\odot$, much larger than the disk masses in the Foucart et al (2014) [cite] study.  The neutrino luminosity peaked $\sym 10$ ms after merger at $L_\nu \sim 10^{54} \textrm{erg} s^{-1}$ and fell to a fifth of this value 50 ms after merger---the disk was quite optically thick in the neutrino sense.  The temperature of the disk and the resulting structure was susceptible to a substantial amount of neutrino cooling, which makes the areal radius of the disk smaller than hotter remnants but still hot and optically thick with non-Keplerian velocities dominating the circulization.

The ejecta had a mass of $M_{\rm ej} \sim 0.08 M_\odot$ with an average velocity $\sim 0.2 c$, although some matter was ejected at $0.5 c$. Additionally, the tail, while under resolved in their study, was very neutron rich with (density averaged) chemical composition $\left\langle Y_e \right\rangle_\textrm{ej} \sim 0.1$  and in nuclear statistically equilibrium as the temperature remained $T_\textrm{ej} \sim 1 \textrm{MeV}$.  The obervable radio transient from the kilonova emmision in the ejecta was estimated to be within the detectable range of the Expanded Very Large Array (EVLA).

% Kyoto group fitiing formulae paper

\textbf{TODO: Kyoto group}

% Foucart et al (2016)

Another recent exploration in the parameter space of hot, microphysical black hole-neutron star systems was done in [cite: focart, dhruv, brege et al (2016)].  In these simulations we used the DD2 equation of state in Section \ref{sec:dd2}, again varying black hole mass $M_{BH} = (5, 7) M_\odot$ and spin $\chi_{BH} = (0.7, 0.9)$ with neutron star mass of $M_{NS} = 1.4 M_\odot$, but this time not restricting the highly prograded spin of the black hole to be orbitally aligned.  The inclination angle was set to $i_{\rm BH}=60^{\circ}$ in all but the high-spin high-mass case where we set $i_{\rm BH}=20^{\circ}$.  We note here that the tidal tail is again much narrower than for the ideal gas or piecewise polytrope equation of state.  

The ejecta was measured to fall in the range of $M_{\rm ej} \sim (0.01 - 0.05) M_\odot$, while higher mass black holes and higher spinning black holes causing the most ejecta.  However, this ejecta masses are much smaller than that of aligned-spin black holes in the previous study [cite: francois 2014]: the magnitude of the orbitally aligned component of the black hole spin is the central parameter in driving ejecta and tidal disruption.  The unbound material was again very neutron rich with $\left\langle Y_e \right\rangle_\textrm{ej} \sim 0.04 - 0.06$, where heavy elements and high-opacity lanthanides are produced, and all of the material has composition $Y_e \le 0.25$, where near this upper limit lighter r-process elements are produced.

$M_{\rm ej} = 0.014 M_\odot$

\textbf{TODO:} More detail about the ejecta needs to go here.

After merger (defined as the time in which half of the baryonic matter has been accreted onto the black hole), the temperature of the fluid reaches a clear local minimum coinciding with a minimizing accretion rate.  Shortly after, the tail begins to self-intersect, causing hydrodynamic shocks that reheat the fluid and causes the bound material to begin circularizing around the black hole.  
This post-merger remnant begins to axisymmetrize $\sim 10$ ms after merger.  
For the M5-S7-I60 system, the disk reached a finite temperature of $3 - 6$ MeV and average electron fraction $Y_e \sim 0.1 - 0.2$.  The disk, still accreting at a rate of $\dot{M}_{BH} \sim 5 M_\odot {\rm s}^{-1}$, has a mass of $M_{\rm disk} = (0.045 - 0.06) M_\odot$ 17 and 10 ms after merger.

These precessing black hole-neutron star simulations were the first general-relativistic systems evolved in which the neutron stars were modeled with a hot, nuclear theory based equation of state and an inclinated black hole spin.

