\chapter{Equation of State Simulations of Black Hole-Neutron Star Mergers}
\label{chap:chapter-4}

In this chapter, we review the literature of numerical relativity in the context of black hole-neutron star mergers, primarily focusing on finite temperature, nuclear theory based Equations of State.  
A robust general relativity code is needed not only to model the gravitational waveform, but also to produce the electromagnetic signals that can be detected by infrared or optical instruments.  
In this sense, the coalescence of black hole neutron star binaries and their mergers are essential systems that can be probed with multi-messenger astronomy.   
There are two essential components of the fluid matter: (1) the tidal tail formed during the merger of the two objects and (2) the remnant accretion disk left after the merger.  
Both of which can produce optically bright outflows.

(1) A large amount of matter in the tail inevitably becomes unbound from the gravitational potential of the black hole.  Matter is said to be ``unbound'' if $u_t < -1$, where $u_alpha$ are the covariant components of the fluid 4-velocity.  
As the disruption causes the unbound matter to be ejected from the system, the neutron-rich ejecta rapidly undergoes radioactive decay in the form of beta and gamma radiation, or ``kilonova'', which can be observed in the near infrared spectrum and on the timescale of a week.
Meanwhile, the gravitational wave signal may only be detectable on the timescale of inspiral and merger ($\sym10$ ms).
The chemical composition of the tail material $Y_e$ provides a measurement of the amount of heavy elements created during the r-process nucleosynthesis of the ejecta. 
Extremely neutron-rich matter ($Y_e \le 0.25$) is expected to produced 2nd and 3rd, but not 1st, r-process elements.  
Their cosmogenic origin of the r-process elements is still uncertain today, so getting a sense of how many of the heavy elements are produced during the merger can help us address this problem.
  
(2)  The accretion disk also contributes to the outflow via disk winds (e.g. by viscous or neutrino absorption heating), although the disk ejecta typically is a factor of a few smaller compared to the tail.
However, disk outflows occur only about 100 ms later than the nucleosynthesis in the tail, and can produce electromagnetic transients in late radio afterglow.
One difference between the two that might be significant is that disk outflows are expected to undergo more neutrino processing and be more neutron starved (higher $Y_e$), which would affect what elements are formed.
Likewise, the polar regions of the disk outflow (jets) will also contribute to the total optical signal output of the event in the form of GRBs.




% Foucart et al (2014)
\section{BHNS Binaries with LS220 Equation of State}

While the inspiral of these systems can be adequately modeled with a simpler, approximate equation of state like a piecewise polytrope, the disruption phase requires a more detailed equation of state with composition and temperature dependence to more fully describe the physical effects.  
To produce the late time gravitational wave signal, when finite size effects of the neutron star are potentially detectable and point mass post-Newtonian approximations begin to have significant errors, one must also evolve the system in the last tens of orbits prior to disruption.  
On the other hand, to fully predict the electromagnetic signal and the nucleosynthesis yields, the number of requisite orbits can be relaxed.  
In addition to the dependencies of neutron star structure (temperature, composition, pressure) to the nuclear reactions, we also have to accurately model magnetic fields effects of the post merger remnant disk and neutrino radiation on the disrupting fluid (and as a significant source of cooling on the disk).

Neutrinos can also play a role on the composition of the unbound material ($u_t < -1$) that gets ejected from the system, as well as the heavy r-process elements created there during disruption. 
Neutrino-antineutrino annihilation in the less dense regions and near the poles of the black hole can contribute to the production of short gamma ray bursts.  
While magnetic fields and the magneto-rotational instability (MRI) are crucial to the late-stage evolution of the accretion disk, simulations have shown that they do not play an essential role during merger.  
The equation of state, however, plays a significant role in all stages of evolution: compared to a stiff equation of state, the relative response of a soft star to the tidal diruption from the black hole during inspiral occurs closer to the black hole (likewise, if at all), can lead to qualitatively different features during merger (thinner stream of matter falling into the black hole), and the multivariate dependence of the equation of state on the evolution of the remnant accretion disk are all essential.

In [cite: foucart, deaton 2014], the first relativistic simulations of black hole-neutron star mergers with a finite temperature equations of state with an accompanying neutrino leakage scheme were performed.  
This work used the Lattimer-Swesty equation of state discussed in Section \ref{sec:ls220}, and explored the parameter space where the mass of the neutron star is in the lower mass range ($M_{\rm NS} = 1.2 M_\odot - 1.4 M_\odot$), while the black hole mass is in the peak of the distribution of likely stellar-mass black holes ($M_{\rm BH} = 7 M_\odot - 10 M_\odot$). 
For these lower-mass black holes, a larger spin is necessary to produce maximal disruption ($\chi_{\rm BH} = 0.7 - 0.9$).  
Their survey was also restricted in the sense that magnetic fields were not included since the disk wasn't evolved for long times (thermal timescales of the disk), noting that the effects of neutrino cooling on the inner regions of the disk and the viscous heating due to the MRI turbulence might more or less offset each other.

Comparing to the piecwise polytropes of Hebeler et al. [cite: 1303.4662], it was found that the use of the hot, nuclear theory based LS220 equation of state was less differentiable during merger for lower mass black holes.  
For higher mass black holes, on the other hand, the difference was most notable as the more rapid disruption occurred later and closer to the black hole with a much thinner tail than with the commonly used (in simulations) $\Gamma = 2$ polytrope (with the same configuration parameters).

There were several limitations noted at the time the simulations were performed.
First, the grid stricture of the finite difference grid was not adaptable at the time.  
That is, given that the tail is much larger in structure than the disk, accurately resolving the accretion disk would lead to over resolving the tail plus evolving enormous amounts of vacuum--far too computationally expensive.
Conversely, lowering resolution by allowing the grid to expand in order to track the ejecta in the tail would drastically under resolve the disk.  
Therefore, to evolve the disk, the large amount of unbound matter had to exit the grid unfortunately early.  Second, a full study of the disk evolution would require the effects of MRI turbulence, or an equivalent effective viscosity routine.
Lastly, the neutrino leakage scheme only goes as far as an estimate for heavy eement nucleosynthesis.
A more robust neutrino transport scheme would be needed to follow the production of neutrino driven winds and electron-positron creation in the baryon-poor polar axes of the black hole (gamma ray bursts).

Key findings from this study found that a finite temperature equation of state neutron stars reliably ejects a large amount of material ($M_\textrm{ej} = 0.04 M_\odot -  0.16 M_\odot$), where less massive stars eject more material, and form accretion disks with mass $M_\textrm{disk} = 0.05 M_\odot -  0.15 M_\odot$, where larger black hole spin lend to more massive disks (and therefore less bound tail material).  The disks are initially hot ($T_\textrm{disk} \sim 5-15 \textrm{MeV}$) and bright in neutrinos ($L_\nu \sim 10^{53} \textrm{erg/s}$).  In the first 10 ms after merger, the disk quickly protonizes from $Y_e \sim 0.06$ to $Y_e \sim 0.1 - 0.4$. 

% Deaton et al (2016)
\subsection{BHNS post-merger remnant study of LS220 Equation of State}

In a follow-up, more in depth study, Deaton et al. [cite: 2016 paper] examined the effects of neutrino cooling on the remnant disk with the configuration system $M_{\rm NS} = 1.4 M_\odot$, $M_{\rm BH} = 5.6 M_\odot$, $\chi_{\rm BH} = 0.9$ for the LS220 equation of state.  
This was the first study of a finite-temperature, composition-dependent equation of state black hole-neutron star system with full general relativity and neutrino cooling.  
Although their disk remnant was only evolved to a fraction of the accretion timescale ($\sym 200$ms), they note that the effects of MRI heating might allow the disk to remain hot long after merger, where magnetic effects can be coupled to the fluid evolution system with magnetohydrodyanmics.


In this study, the post merger remnant had baryon rest mass $M_\textrm{disk} \sim 0.3 M_\odot$, much larger than the disk masses in the Foucart et al (2014) [cite] study.  
This is to be expected because of the smaller black hole mass.
The neutrino luminosity peaked $\sym 10$ ms after merger at $L_\nu \sim 10^{54} \textrm{erg} s^{-1}$ and fell to a fifth of this value 50 ms after merger---the disk was quite optically thick in the neutrino sense.  The temperature of the disk and the resulting structure was susceptible to a substantial amount of neutrino cooling, which makes the areal radius of the disk smaller than hotter remnants but still hot and optically thick with non-Keplerian velocities dominating the circulization.

The ejecta had a mass of $M_{\rm ej} \sim 0.08 M_\odot$ with an average velocity $\sim 0.2 c$, although some matter was ejected at $0.5 c$. Additionally, the tail, while under resolved in their study, was very neutron rich with (density averaged) chemical composition $\left\langle Y_e \right\rangle_\textrm{ej} \sim 0.1$  and in nuclear statistically equilibrium as the temperature remained $T_\textrm{ej} \sim 1 \textrm{MeV}$.  The obervable radio transient from the kilonova emmision in the ejecta was estimated to be within the detectable range of the Expanded Very Large Array (EVLA).

% Kyoto group fitiing formulae paper
\section{BHNS binaries with zero-temperature, piecewise polytrope EOS}

We take a quick pause from hot nuclear equations of state here to discuss the findings of the Kyoto group where mixed binaries are studied with a large swath of piecewise polytrope zero-temperature equations of state with the adaptive mesh refinement numerical relativity code, \SACRA (SimulAtor for Compact objects in Relativistic Astrophysics).  
As mentioned in Chapter \ref{chap:chapter-2}, the utility of a PP equation of state is that it provides an approximation to realistic equations of state, making it easier to systematically explore a small number of different parameter estimates for the core and crust material of the star.  

In [cite: 1502.05402], four different equations of state with a wide range of compactnesses $\mathcal{C} = (0.138 - 0.180)$ were studied with black hole spin $\chi = (0, 0.5, 0.75)$, and mass ratio $q \equiv M_{\rm BH}/M_{\rm NS} = (3, 5, 7)$ with the initial configuration for the star always assuming an ADM mass of $M_{\rm NS} = 1.35 M_\odot$.
This survey led to the estimation of an updated analytic, fitting formula to determine the mass and velocity of the dynamical ejecta [cite: 1601.07711].  They obtained:
\begin{align}
\label{e:vej_sacra}
\left\langle v_{\rm ej} \right\rangle = (0.01533 q + 0.1907) c
\end{align}
An analytic model for the light curve of the resulting kilonova/macronova is also determined.
Prior to this study, Foucart [cite: 1207.6304] determined a simple formula to relate the dimensionless parameters of the initial configuration system ($\mathcal{C}_{\rm NS}, q, \chi_{\rm BH} $)  with the mass of the post-merger remnant.

% Foucart et al (2016)

\section{Precessing BHNS binaries with hot, microphysical EOS}

Another recent exploration in the parameter space of hot, microphysical black hole-neutron star systems was done in [cite: focart, dhruv, brege et al (2016)].
This was the first time such mergers were studied in the parameter space in which the black hole spin was inclined with respect to the orbital angular momentum. 
In these simulations we used the DD2 equation of state (cf. Section \ref{sec:dd2}), again varying black hole mass $M_{\rm BH} = (5, 7) M_\odot$ and spin $\chi_{\rm BH} = (0.7, 0.9)$ with neutron star mass of $M_{\rm NS} = 1.4 M_\odot$, but this time not restricting the highly prograded spin of the black hole to be orbitally aligned.  The inclination angle was set to $i_{\rm BH}=60^{\circ}$ in all but the high-spin high-mass case where we set $i_{\rm BH}=20^{\circ}$.  We note here that the tidal tail is again much narrower than for the ideal gas or piecewise polytrope equation of state.  

The ejecta was measured to fall in the range of $M_{\rm ej} \sim (0.01 - 0.05) M_\odot$, while higher mass black holes and higher spinning black holes causing the most ejecta.  However, this ejecta masses are much smaller than that of aligned-spin black holes in the previous study [cite: Francois 2014]: the magnitude of the orbitally aligned component of the black hole spin is the central parameter in driving ejecta and tidal disruption.  The unbound material was again very neutron rich with $\left\langle Y_e \right\rangle_\textrm{ej} \sim 0.04 - 0.06$, where heavy elements and high-opacity lanthanides are produced, and all of the material has composition $Y_e \le 0.25$, where near this upper limit lighter r-process elements are produced.  The mass of the ejecta was $M_{\rm ej} = 0.014 M_\odot$.

Comparing to the fitting formulae in Equation \ref{e:vej_sacra}, the results of the simulations with the hot, nuclear-theory based equation of state were quite different.  For the $5 M_{\rm BH}$ black hole system, one would expect that 
$\left\langle v_{\rm ej, rms} \right\rangle = 0.245 c$.  The result in the \SpEC simulations gave $\left\langle v_{\rm ej, rms} \right\rangle = 0.175 c$.  To account for the $~40 \%$ difference between the two results, one has to keep in mind that the fitting formula of  Equation \ref{e:vej_sacra} was based on the use of zero-temperature equations of state, no composition dependence and bet-equilibrium.  The finite-temperature DD2 equation of state, on the other hand, only assumes muclear statistic equilibrium and is composition dependent.  

\textbf{Section 4 of the paper (might not be worth explaining): }

\textbf{TODO * }There is a physical difference: the effective chemical composition for the cold equation of state at low-densities is $Y_e \sim 0.5$ (mostly $^{56}{\rm Ni}$) , while the Rosswog leakage calculation on the DD2 equation of state simulations produced $Y_e \sim 0.05$ (mostly neutrons and $\sim 10\%$ seed nuclei at cold temperatures).
If all of the energy required to transform from neutron-rich ejecta to $^{56}{\rm Ni}$ was turned into kinetic energy, the simulations would produce ejecta with average velocity $\left\langle v_{\rm ej, rms} \right\rangle = 0.22 c$, within $\sim 10\%$ of the fitting formula prediction.
If half of the energy released is due beta-decays producing relativistic electrons and neutrinos, $\sym 4.6 {\rm MeV/nuc}$ of kinetic energy can be accounted for in comparison to the piecewise polytrope models.
We get the fitting formula:
\begin{align}
\label{e:vej_spec}
\left\langle v_{\rm ej, rms} \right\rangle = (0.0166 q + 0.1667) c
\end{align}
 \textbf{ * }


After merger (where half of the baryonic matter has already been accreted onto the black hole), the temperature of the fluid reaches a clear local minimum, coinciding with a minimizing accretion rate.  Shortly after, the tail begins to self-intersect, causing hydrodynamic shocks that reheat the fluid and causes the bound material to begin circularizing around the black hole.  
This protodisk begins to axisymmetrize $\sim 10$ ms after merger.  
For the M5-S7-I60 system, the disk reached a finite average temperature of $3 - 6$ MeV and average electron fraction $Y_e \sim 0.1 - 0.2$.  The disk, still accreting at a rate of $\dot{M}_{\rm BH} \sim 5 M_\odot {\rm s}^{-1}$, has a mass of $M_{\rm disk} = (0.045 - 0.06) M_\odot$ 17 and 10 ms after merger.


\textbf{NOTES: What else to include in this review?}  