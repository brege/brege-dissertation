\chapter{Overview}
\label{chap:chapter-1}


Black holes and neutron stars are the most compact objects in the known universe.  
The mergers of these two objects,
 whether mixed in the form of black hole-neutron star (BHNS) systems or alike in the form of binary black hole (BBH) and neutron star-neutron star (NSNS) systems,
 are a primary source of gravitational waves.  
These objects form from super-massive stars,
 with masses $M\gtrsim 8\,M_{\odot}$,
 whose cores collapse from having burnt all their fissile material and thereby
 losing essential support from gas pressure.
Neutron stars result from supernova on the smaller-mass end of this population,
 and are supported against gravitational collapse by degenerate neutron pressure,
 not by burning fuel,
 and with masses ranging between $1-2\,M_{\odot}$.
To the left,
 smaller mass stars collapse into white dwarfs,
 who are supported by degenerate electrons.  
Fermi pressure from an electron gas is not as strong as that of a neutron star,
 so white dwarfs cannot be supported unless they have masses $M\lesssim 1.4\,M_{\odot}$,
 known as the Chandrasekhar limit.  
Lastly, for masses larger than $M\gtrsim 10\,M_{\odot}$,
 gas pressure of any kind cannot sustain an equilibrium with gravitational pressure.  
These collapses result in stellar-massive black holes.

Neutron stars are interesting species because during their formation for supernovae,
 they collapse to a density close to that of nuclei.
Objects of such extreme densities have areal radii of $\sim 10 \textrm{km}$,
 smaller than the distance between Pullman, WA and Moscow, ID.
While dangerous, you could bike around an area containing the entire mass of our solar system (plus some) in a single day. 
Quantitatively, the compactness of an object is a measure of the how much mass is within an enclosed surface.
Compactness is unitless and defined by $\mathcal{C}\equiv G M/c^2 R$.
The factor $G M/c^2$ in the compactness paramater is the \textit{Schwartzchild radius},
 which when comparable to a stellar radius means general relativity plays an important role.
For reference,
% the compactness of the earth and sun are
% $\mathcal{C}_\textrm{E} \sim 10^{-10}$
% and
% $\mathcal{C}_{\odot} \sim 10^{-6}$,
 the Sun, TRAPPIST-1 and a typical white dwarf are
 $\mathcal{C}_{\odot} \approx 4 \times 10^{-6}$,
 $\mathcal{C}_\textrm{T-1} \approx 0.7\,\mathcal{C}_{\odot}$
 and
 $\mathcal{C}_\textrm{WD} \approx 1.8 \times 10^{-4}$,
 respectively.
Clearly general relativity does not play a major role in these stars.
Meanwhile, for neutron stars, and black holes, the compactnesses are 
 $\mathcal{C}_\textrm{NS} \approx 0.15$
 and
 $\mathcal{C}_\textrm{BH} = 0.50$.
At this end of the compaction scale, general relativity cannot be ignored.

To further get a sense for how dense neutron stars are, a quick back of the envelope calculation is to consider a $M_\textrm{NS} = 1.4 M _\odot$ star, which when divided by the mass of a neutron accounts for $\sym 1.7 \times 10^{57}$ total nucleons.
If the nucleons are homogeneously dispersed in this star, with areal radius of $12\,\textrm{km}$, the separation between them is then $\sym 1\,\textrm{fm}$.
From scattering experiments, we know the spacing of nucleons to be $\sym 0.8\,\textrm{fm}$ so the baryons are essentially touching.
In this compactness regime, we expect all fundamental forces to be important.
The gas pressure, especially in the inner and outer core, is largely dominated by the strong force.  
The crust is mostly supported by coulomb forces which in equilibrium yield lattice structures.
We will also talk discuss $\beta$ equilibrium in a Chap. \ref{chap:chapter-2}, which is governed by the weak force.

Due to the conservation of angular momentum, neutron stars---much smaller in breadth than their mothers---have very high spins.

Magnetized neutron stars with extreme spins, or pulsars that emit X-rays,
 have received immense interest within the community.

But the focus of this dissertation is not on that of neutron stars in solitary,
 but their coupling with black holes and their violent, eventual merger.

