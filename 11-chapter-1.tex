\chapter{Overview}
\label{chap:chapter-1}


Black holes and neutron stars are the most compact objects in the known universe.  
The mergers of these two objects,
 whether mixed in the form of black hole-neutron star (BHNS) systems or alike in the form of binary black hole (BBH) and neutron star-neutron star (NSNS) systems,
 are a primary source of gravitational waves.  
These objects form from super-massive stars,
 with masses $M\gtrsim 8\,M_{\odot}$,
 whose cores collapse from having burnt all their fissile material and thereby
 losing essential support from gas pressure.
Neutron stars result from supernova on the smaller-mass end of this population,
 and are supported against gravitational collapse by degenerate neutron pressure,
 not by burning fuel,
 and with masses ranging between $1-2\,M_{\odot}$.
To the left,
 smaller mass stars collapse into white dwarfs,
 who are supported by degenerate electrons.  
Fermi pressure from an electron gas is not as strong as that of a neutron star,
 so white dwarfs cannot be supported unless they have masses $M\lesssim 1.4\,M_{\odot}$,
 known as the Chandrasekhar limit.  
Lastly, for masses larger than $M\gtrsim 10\,M_{\odot}$,
 gas pressure of any kind cannot sustain an equilibrium with gravitational pressure.  
These collapses result in stellar-massive black holes.

Neutron stars are interesting species because during their formation,
 they collapse to a density close to that of nuclei.
Objects of such extreme densities have areal radii of $\sim 10 \textrm{km}$,
 smaller than the distance between Pullman, WA and Moscow, ID.
While dangerous, you could bike around an area containing the entire mass of our solar system (plus some) in a single day. 
Quantitatively, the compactness of an object is a measure of the how much mass is within an enclosed surface.
Compactness is unitless and defined by $\mathcal{C}\equiv G M/c^2 R$.
For reference,
% the compactness of the earth and sun are
% $\mathcal{C}_\textrm{E} \sim 10^{-10}$
% and
% $\mathcal{C}_{\odot} \sim 10^{-6}$,
 the Sun and TRAPPIST-1 are
 $\mathcal{C}_{\odot} \sim 10^{-6}$
 and
 $\mathcal{C}_\textrm{T-1} \sim 0.7 \mathcal{C}_{\odot}$
 respectively.
For white dwarfs, neutron stars, and black holes, the compactnesses are 
 $\mathcal{C}_\textrm{WD} \sim 10^{-4}$,
 $\mathcal{C}_\textrm{NS} \sim 0.15$,
 and
 $\mathcal{C}_\textrm{BH} = 0.50$.
At this end of the compaction scale, general relativity plays a major role.
Due to the conservation of angular momentum, neutron stars---much smaller in breadth than their mothers---have very high spins.

Magnetized neutron stars with extreme spins, or pulsars that emit X-rays,
 have received immense interest within the community.

But the focus of this dissertation is not on that of neutron stars in solitary,
 but their coupling with black holes and their violent, eventual merger.

