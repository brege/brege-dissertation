\chapter{Neutron Star Equation of State}
\label{chap:chapter-2}

In this chapter, we take a heuristic approach in building up to the rather complex modern prescriptions of neutron star structure. 
Astrophysicists have posited still-relevant models of stellar structure before the frameworks of general relativity and quantum mechanics were even known, mostly forming their understanding fundamentally with thermodynamical and statistical arguments and an as-yet unchallenged Newtonian description of gravity.
This plan is also used in hindsight.
As a description of a system gets more complicated (finite-temperature nuclear physics, indirect extrapolations of observational and experimental data), it is often necessary to ground our understanding with an approximate model.
Here, the simple model for the equation of state is that of a polytrope, or one where the pressure is only dependent on the density of the fluid or gas:
\begin{align}
P(\rho_0) = K \rho_0^\Gamma = (\Gamma -  1) \rho_0 \epsilon,
\end{align}
where $P$ is pressure, $\epsilon$ specific energy, $\rho_0$ rest-mass density, $K$ the gas constant, and $\Gamma$ is a unitless exponent often referred to as the adiabatic index.
Since the density only varies with the pressure, this model assumes the fluid is barotropic.

This simple model reveals itself to be important in limiting cases of stellar matter.  
For a high-mass neutron star where the average density is well above the nuclear saturation number density $n_s$, the star can be adequately described with a $\Gamma \approx 2$ polytrope.  
For a low-mass neutron star, where the average density is well below $n_s$, the star is mostly supported by degenerate Fermi pressure (as is in the white dwarf model) such that $\Gamma \simeq 4/3$.
Introducing the notion that the interior region of the star may have a different equilibrium configuration to the crust, and that the core could be composed of exotic matter, a more adaptable equation of state can be constructed with a set of polytropes each defining these regions.  
These equations of state are called piecewise-polytropes.  

However, the polytrope models do not adequately account for effects of temperature or composition elements of our very dense fluid matter.  
So called ``hot'' or finite-temperature equations of state are much more nuanced, and generally constructed as multivariable tables utilizing many constraints from nuclear theory, astrophysics observations, and particle experiments.
Because of the complexities for codes to utilize these equations of state, piecewise-polytropes are often used as a way to curve-fit and cheat the incorporation of temperature in their routines, as well as to provide a simple method to tune parameters of the PP to explore a wide range of neutron star compactnesses.
In low-temperature bins of the fluid, a simple ``Gamma-Law'' is often used to smoothly transition from zero-temperature to the lower limit of some nuclear-theory based equation of state tables.
In addition, an effective adiabatic index can be extracted to give one a sense how stiff the structure is in a given density regime:
\begin{align}
\label{e:adiabat}
\tilde{\Gamma} \equiv \left( \frac{d\, {\rm log} P}{d\, {\rm log} \rho_0} \right) _S
\end{align}

Many of these equations of state are further supported by simulations of core-collapse supernovae.

First, we'll review the theory of polytrope based models in Section \ref{sec:polytropes}. Then we will outline the many experimental, observational and nuclear theory constraints on the equation of state in Section \ref{sec:constraints}.   In Section \ref{sec:nuclear-eos}, we include a description of the microphysical equations of state used in the simulations of Chapter \ref{chap:chapter-5}. 


\section{Polytropes}
\label{sec:polytropes}

Neutron stars were initially modeled with the polytrope, or ``Gamma-Law'', equation of state, in the range were $2 < \Gamma < 3$.
For higher $\Gamma$'s, the gas pressure is stronger, and we often refer to these as ``stiff'' equations of state and look poofier, or less compact.
Conversely, stars with smaller $\Gamma$'s are called ``soft''.  Soft equations of state are hence more compact.

We do not expect, further, that the core, interior or crust to all have the same polytropic description.
In that case, the Gamma-Law description of neutron star structure can be improved by a so-called piecewise polytrope model.
In such models, we allow each regime to have its own equation of state such that 
$$P(\rho_0) = K_i \rho_0^{\Gamma_i}$$
where the adiabatic index $\Gamma_i$ is a constant in each regime, and $K_i$ is chosen so that the equation of state is continuous across interfaces.  
Piecewise polytropes, while multi-parameter, are still single-variable descriptions of stellar structure.  
A first method of adding thermal dependencies to any polytrope model is to augment the ``cold'' equation of state:
\begin{align}
P &=  P_{\rm cold}(\rho_0) + (\Gamma_{\rm th}-1)\rho\epsilon_{\rm th} ,\\
\epsilon &=  \epsilon_{\rm cold}(\rho_0) + \epsilon_{\rm th}, 
\end{align}
where $\epsilon_{\rm th}$ is a thermal specific energy, which is determined during the evolution of the energy density.

\textbf{TODO: explain PP + GW detection and PP study by lackey+kyoto group}


In addition to the density dependence of pressure, a more realistic equation of state has to include dependence on temperature, $T$, and composition, for instance the fraction of electrons per baryons $Y_e = n_e / n_b$.  
We expect the temperature to be a factor, since hydrodynamical shocks heat the system, and the composition to vary since neutrons tend to decay and neutrinos allow for cooling or further heating. 
In general, this equation of state system takes the form:
\begin{align}
P &= P(\rho_0, T, X_i), \\
u &= u(\rho_0, T, X_i),
\end{align}
where $X_i$ can be any number of composition variables.

\section{Neutron Star Constraints}
\label{sec:constraints}

\subsection{Observational Constraints}

Results from X-ray binary observations constrain the areal radius of $1.4M_\odot$ neutron stars between $10.4 - 12.9 \textrm{km}$.
Observations from the neutron star-white dwarf binary PSR J1614-2230 revealed a neutron star mass $1.97 \pm 0.04 M_\odot$, and later measurements from PSR J0348+0432 showed masses slightly higher at  $2.01 \pm 0.04 M_\odot$.  Therefore, an equation of state must allow for a maximal mass of at least $2.05 M_{\odot}$.

\subsection{Experimental Constraints}

The most rapidly spinning pulsar PSR J1748-2446ad constrains the maximum radius of a neutron star: a star must rotate beneath the Keplerian frequency at which mass-shedding occurs [cite:Latt. 2015 eq 95].

\textbf{TODO}: Use arguments from 1403.1186 which give general experimental constraints that all NS EOS need to satisfy. 

\subsection{Theoretical Constraints}

From \textit{Le Chatlier's principle}, the maximum mass of a neutron star can be no larger than 3.2 $M_\odot$.  Lower bounds on the neutron star radius can be obtained through causality, namely that the sound speed $c_s^2 \equiv c^2 \partial p / \partial \epsilon$ must not be superluminal (i.e. $c_s^2 \le c^2$).  

\section{Hot, Microphysical Equations of State}
\label{sec:nuclear-eos}

\subsection{LS220}
\label{sec:ls220}

One of the first finite-temperature equations of state developed is the Lattimer-Swesty (LS) model.  For this reason, computationally expensive full, three-dimensional fluid evolutions of core-collapse supernova have only been performed for LS220 and STOS [\textbf{TODO}: fact check this].
LS is based on a compressible liquid drop model, using a first-order phase transition from low-density vapor to the high-density liquid phase.  This makes the density dependence of the transition isobaric.  The incompressibility parameter is $K = 220 {\rm MeV}$.

\subsection{DD2}
\label{sec:dd2}

Another hot, composition dependent nuclear-theory based equation of state is DD2.  DD2 is relativistic mean field (RMF) model based on a nucleon-meson coupling model whose momentum is density dependent (DD).  
For DD, the reference density is the saturation density (when nucleons begin to touch) of symmetric nuclear matter (i.e. the number densities $n_\textrm{p} = n_\textrm{n}$), where the incompressibility factor is fixed at $K = 240$ MeV.
In this model, the binding energy of eight nuclei were computed and compared to experimental values and found stronger agreement than the NL3 EOS.
DD2 is the same as DD with additional corrections provided from experimental nucleon masses measured in a lab.

DD2 is also constrained by several nuclear theory based studies.
Several studies have shown that nuclear energy bounds fall within the predictions of Chiral Effective Field theory.
Unlike the classical Lattimer-Swesty [cite] and the STOS [cite] EOS models that employ a single nucleus approximation, DD2 includes a detailed distribution of thousands of different nuclei.

In addition, DD2 allows for the existence of stars whose masses are above the observations we have to-date.
Compared to observational constraints, DD2 can allow for masses as high as $\sym 2.4 M_\odot$, much greater than the observed $\sym 2 M_\odot$ neutron stars.
DD2, however, comes within $\sym 1 \textrm{km}$ of measured areal radii--being just slightly outside of with $13.22 \textrm{km}$, but within statistical uncertainties.

\subsection{FSU2.1}
\label{sec:fsu21}

Another relativistic mean field equation of state is one based on the FSUGold model with modifications in the high density bins to allow for a maximal neutron star mass of $2.1 M_\odot$.  The RMF effective interaction FSUGold model only predicts a $1.7 M_\odot$ maximum mass, and was constructed before the detection of the $1.97 \pm 0.04 M_\odot$ neutron star.
Among the assumptions and constraints that went into the FSUGold equation of state, FSU2.1 corrects the pressure approximately independent of proton fraction, and likewise independent of the temperature, as matter is degenerate at high densities.

The upper limit on the baryon number density $n_B$ was modified from $10^{0.4} {\rm fm^{-3}}$ to $10^{0.2} {\rm fm^{-3}}$ for finite temperatures $T \in [10^{-0.8} , 10^{1.875}]\, {\rm MeV}$.  The lower limit is the same as FSU1.7 at $10^{-8.0} {\rm fm^{-3}}$.  For zero-temperature, the proton fraction $Y_p$ is held at $0$ while in the finite-temperature regime, $Y_p \in [0.05 , 0.56]$.  The incompressibility parameter is $K = 230.0 MeV$

\subsection{SFHo and SFHx}
\label{sec:sfh}

Two additional systems were developed by Steiner, Hempel and Fischer that yielded much more compact stars.  In these models, the theoretical constraint enforcing subluminal sound speeds is automatically enforced.  A prediction for the region where the $M-R$ curve of a $1.4 M_\odot$  neutron star is given in [cite:1205.6871], where the authors argue that the radius should be $R_{\rm NS} \in [11.2,12.3] {\rm km}$.  The baseline model, SFHo, fits this prediction.  The extreme model, SFHx, attempts to minimize the radius of lower mass neutron stars.  In either case, the pressure monotonically increases with density in all phases.
The incompressibility factor is $K = 245.4 {\rm MeV}$ for SFHo and $K = 238.8 {\rm MeV}$ for SFHx

\section{Our survey}

In our survey, we use the Hempel's DD2, Fisher's FSU2.1, and the SFHo and SFHx equations of state.  LS220 may occasionally be used as a reference. 
In Figure \ref{fig:MvsR}, we show the neutron star mass-radius relationship for each system used in this study.  We chose to study two neutron star masses, $M_{\rm NS} = (1.2, 1.4) M_{\odot}$, since the distribution of neutron star populations is expected to have a mean in this range. In addition, the LS220 equation of state is included, as it has already been simulated in [cite:Foucart et al (2014)].  For the initial configuration of the star in isolation, we take a cold, $\beta$-equilibrium slice of three-dimensional tables.  To avoid strange artifacts at the table minimum of $T_{\rm floor} = 0.01 {\rm MeV}$, we take a cold slice of  $T = 0.1 {\rm MeV}$.  Therefore, the effective (cold, $\beta$-equilibrium) equation of state table is barotropic. Once the star begins to tidally disrupt, we are no longer in $\beta$-equilibrium.

In [cite:Lattimer 2015], Lattimer discusses the number of constraints on the equation of state and how they relate to the mass-radius relations in various ``zones'' of Figure \ref{fig:MvsR}.  Principally, an equation of state must not allow for superluminal sound speeds and allow for a maximum mass of $\sym 2.1 M_{\odot}$.

\begin{figure}
	\centering
	\input{images/tov-mass-vs-radius}
	\caption[Neutron star mass vs. areal radius]{
		ADM neutron star mass vs. areal radius for nuclear equations of state sliced at $T=0.1{\rm MeV}$ in $\beta$-equilibrium.  Intersections of dotted lines and colored curve represent the neutron star masses chosen for this survey: $M_{\rm NS} = (1.2, 1.4) M_{\odot}$.  
	}
	\label{fig:MvsR}
\end{figure}

From the $M-R$ curves in Figure \ref{fig:MvsR} alone, there are a bounty of astrophysics questions we can ask.  By adopting these finite-temperature, composition-dependent equation of state models for our neutron stars, what observable characterisitics are observed during inspiral and merger?  How much does the creation of r-process elements in the dynamical outflows vary?  Does the generated gravitational waveform vary outside of the compactness parameter?  Are the fitting formulae predictions of the resulting disk and ejecta masses correct, even though they were formed with barotropic equations of state?

\begin{figure}
	\centering
	% GNUPLOT: LaTeX picture with Postscript
\begingroup
  \makeatletter
  \providecommand\color[2][]{%
    \GenericError{(gnuplot) \space\space\space\@spaces}{%
      Package color not loaded in conjunction with
      terminal option `colourtext'%
    }{See the gnuplot documentation for explanation.%
    }{Either use 'blacktext' in gnuplot or load the package
      color.sty in LaTeX.}%
    \renewcommand\color[2][]{}%
  }%
  \providecommand\includegraphics[2][]{%
    \GenericError{(gnuplot) \space\space\space\@spaces}{%
      Package graphicx or graphics not loaded%
    }{See the gnuplot documentation for explanation.%
    }{The gnuplot epslatex terminal needs graphicx.sty or graphics.sty.}%
    \renewcommand\includegraphics[2][]{}%
  }%
  \providecommand\rotatebox[2]{#2}%
  \@ifundefined{ifGPcolor}{%
    \newif\ifGPcolor
    \GPcolortrue
  }{}%
  \@ifundefined{ifGPblacktext}{%
    \newif\ifGPblacktext
    \GPblacktexttrue
  }{}%
  % define a \g@addto@macro without @ in the name:
  \let\gplgaddtomacro\g@addto@macro
  % define empty templates for all commands taking text:
  \gdef\gplbacktext{}%
  \gdef\gplfronttext{}%
  \makeatother
  \ifGPblacktext
    % no textcolor at all
    \def\colorrgb#1{}%
    \def\colorgray#1{}%
  \else
    % gray or color?
    \ifGPcolor
      \def\colorrgb#1{\color[rgb]{#1}}%
      \def\colorgray#1{\color[gray]{#1}}%
      \expandafter\def\csname LTw\endcsname{\color{white}}%
      \expandafter\def\csname LTb\endcsname{\color{black}}%
      \expandafter\def\csname LTa\endcsname{\color{black}}%
      \expandafter\def\csname LT0\endcsname{\color[rgb]{1,0,0}}%
      \expandafter\def\csname LT1\endcsname{\color[rgb]{0,1,0}}%
      \expandafter\def\csname LT2\endcsname{\color[rgb]{0,0,1}}%
      \expandafter\def\csname LT3\endcsname{\color[rgb]{1,0,1}}%
      \expandafter\def\csname LT4\endcsname{\color[rgb]{0,1,1}}%
      \expandafter\def\csname LT5\endcsname{\color[rgb]{1,1,0}}%
      \expandafter\def\csname LT6\endcsname{\color[rgb]{0,0,0}}%
      \expandafter\def\csname LT7\endcsname{\color[rgb]{1,0.3,0}}%
      \expandafter\def\csname LT8\endcsname{\color[rgb]{0.5,0.5,0.5}}%
    \else
      % gray
      \def\colorrgb#1{\color{black}}%
      \def\colorgray#1{\color[gray]{#1}}%
      \expandafter\def\csname LTw\endcsname{\color{white}}%
      \expandafter\def\csname LTb\endcsname{\color{black}}%
      \expandafter\def\csname LTa\endcsname{\color{black}}%
      \expandafter\def\csname LT0\endcsname{\color{black}}%
      \expandafter\def\csname LT1\endcsname{\color{black}}%
      \expandafter\def\csname LT2\endcsname{\color{black}}%
      \expandafter\def\csname LT3\endcsname{\color{black}}%
      \expandafter\def\csname LT4\endcsname{\color{black}}%
      \expandafter\def\csname LT5\endcsname{\color{black}}%
      \expandafter\def\csname LT6\endcsname{\color{black}}%
      \expandafter\def\csname LT7\endcsname{\color{black}}%
      \expandafter\def\csname LT8\endcsname{\color{black}}%
    \fi
  \fi
  \setlength{\unitlength}{0.0500bp}%
  \begin{picture}(7200.00,5040.00)%
    \gplgaddtomacro\gplbacktext{%
      \csname LTb\endcsname%
      \put(1100,758){\makebox(0,0)[r]{\strut{} 1e+27}}%
      \put(1100,1220){\makebox(0,0)[r]{\strut{} 1e+28}}%
      \put(1100,1682){\makebox(0,0)[r]{\strut{} 1e+29}}%
      \put(1100,2145){\makebox(0,0)[r]{\strut{} 1e+30}}%
      \put(1100,2607){\makebox(0,0)[r]{\strut{} 1e+31}}%
      \put(1100,3069){\makebox(0,0)[r]{\strut{} 1e+32}}%
      \put(1100,3531){\makebox(0,0)[r]{\strut{} 1e+33}}%
      \put(1100,3993){\makebox(0,0)[r]{\strut{} 1e+34}}%
      \put(1100,4455){\makebox(0,0)[r]{\strut{} 1e+35}}%
      \put(1455,440){\makebox(0,0){\strut{} 1e+11}}%
      \put(2579,440){\makebox(0,0){\strut{} 1e+12}}%
      \put(3703,440){\makebox(0,0){\strut{} 1e+13}}%
      \put(4827,440){\makebox(0,0){\strut{} 1e+14}}%
      \put(5950,440){\makebox(0,0){\strut{} 1e+15}}%
      \put(160,2719){\rotatebox{-270}{\makebox(0,0){\strut{}Pressure (barye)}}}%
      \put(4029,140){\makebox(0,0){\strut{}Density (g/cm$^3$)}}%
    }%
    \gplgaddtomacro\gplfronttext{%
      \csname LTb\endcsname%
      \put(3140,4636){\makebox(0,0)[r]{\strut{}Hempel DD2}}%
      \csname LTb\endcsname%
      \put(3140,4436){\makebox(0,0)[r]{\strut{}G. Shen FSU 2.1}}%
      \csname LTb\endcsname%
      \put(3140,4236){\makebox(0,0)[r]{\strut{}SFHo}}%
      \csname LTb\endcsname%
      \put(3140,4036){\makebox(0,0)[r]{\strut{}SFHx}}%
      \csname LTb\endcsname%
      \put(3140,3836){\makebox(0,0)[r]{\strut{}LS220}}%
    }%
    \gplbacktext
    \put(0,0){\includegraphics{images/pressure-vs-density}}%
    \gplfronttext
  \end{picture}%
\endgroup

	\caption[Pressure vs. density for a cold, beta-equilibrium slice]{
		Pressure vs. density for finite-temperature, nuclear equations of state used in this study. The initial fluid is chosen to be cold and in $\beta$-equilibrium: $T=0.1 {\rm MeV}$ and $Y_e = Y_e (n_b)$, respectively.  Vertical dotted line represents the fiducial density.
	}
	\label{fig:PvsRho}
\end{figure}

In Figure \ref{fig:PvsRho}, it can be seen that the relationship between pressure and density at low densities is similar to a $\Gamma \approx 2$ polytrope in the ``cold'' regime of the table.  Qualitatively, the pressure support at lower densities can be stronger for one equation of state, but softer at higher densities compared to another.  Indeed, it was noted in [cite: SFHo paper] that the concept of describing an equation of state as stiff or soft can be na\"{i}ve.  That is, the effective adiabatic index $\tilde{\Gamma}$ of Equation \ref{e:adiabat} for the poofier FSU2.1 model is smaller (larger) in the lower (higher) density regions compared to the more compact SFHo in the ``cold'' phase.

\begin{figure}
	\centering
	\input{images/delta-pressure-vs-density}
	\caption[]{
		Normalized pressure difference at $Y_e = 0.05$ between $T = (0,5) {\rm MeV}$ (solid) and $T = (0,10) {\rm MeV}$ (dashed), both normalized by the zero-temperature pressure. Vertical dotted line represents the fiducial density.
	}
	\label{fig:dPvsRho}
\end{figure}

In Figure \ref{fig:dPvsRho}, we observe the effect temperature has on the pressure-density relations.  Past the fiducial density $\rho_s = 10^{14.7} {\rm g \, cm^{-3}}$ (with fixed $Y_e = 0.05$), changes in temperature do not alter the pressure-density relation by much at all.  As temperature increases in the lower density regime, the structures begin to converge and increase to similar $\tilde{\Gamma}$ between models.  The main differences occur in the intermediate region, where the increasing variance in $\tilde{\Gamma}$ between temperatures becomes clear.