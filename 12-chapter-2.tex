\chapter{Neutron Star Equation of State}
\label{chap:chapter-2}

We begin this chapter with a brief primer on the classical Equation of State of stars.
A polytrope is a solution to the Lane-Emden equation of the form
$$P(\rho) = K \rho^\Gamma$$
where $P$ is pressure, $\rho$ rest-mass density, $K$ the gas constant, and $\Gamma$ is a unitless exponent often referred to as the adiabatic index.
Neutron stars have been fairly well modeled with this ``Gamma-Law'' in the range $2 < \Gamma < 3$.
For higher $\Gamma$'s, the gas pressure is stronger, and we often refer to these as ``stiff'' equations of state and are poofier.
Conversely, stars with smaller $\Gamma$'s are called ``soft''.  Soft equations of state are hence more compact.

We do not expect, further, that the core, interior or crust to all have the same polytropic description.
In that case, the Gamma-Law description of neutron star structure can be improved by a so-called piecewise polytrope model.
In such models, we allow each regime to have its own equation of state such that 
$$P(\rho) = K_i \rho^{\Gamma_i}$$
where the adiabatic index $\Gamma_i$ is constant in each regime, and $K_i$ is chosen so that the equation of state is continuous across interfaces.

\textbf{TODO}: 
\begin{itemize}
	\item add description of typicl polytrope construction  (0812.2163)
	\item explain how PP's and different $\Gamma$'s can be detected by LIGO (1109.3402)
\end{itemize}

Piecewise polytropes, while multi-parameter, are still single variable descriptions of stellar structure.  We require that the pressure of a super-dense hot fluid can depend on the temperature, neutrino luminosity and composition of the matter within it.

\section{Neutron Star Constraints}

\subsection{Observational Constraints}

Results from X-ray binary observations constrain the areal radius of $1.4M_\odot$ neutron stars between $10.4 - 12.9 \textrm{km}$.
Observations from the neutron star-white dwarf binary PSR J1614-2230 revealed a neutron star mass $1.97 \pm 0.04 M_\odot$, and later measurements from PSR J0348+0432 showed masses slightly higher at  $2.01 \pm 0.04 M_\odot$.

\subsection{Experimental Constraints}

\textbf{TODO}: Use arguments from 1403.1186 which give general experimental constraints that all NS EOS need to satisfy. 

\section{LS220}

$K = 220$ MeV

\section{DD2}

Another hot, composition dependent nuclear-theory based equation of state is DD2.  DD2 is relativistic mean field model based on a nucleon-meson coupling model whose momentum is density dependent (DD).  
For DD, the reference density is the saturation density (when nucleons begin to touch) of symmetric nuclear matter (i.e. the number densities $n_\textrm{p} = n_\textrm{n}$), where the incompressibility factor is fixed at $K = 240$ MeV.
In this model, the binding energy of eight nuclei were computed and compared to experimental values and found stronger agreement than the NL3 EOS.
DD2 is the same as DD with additional corrections provided from experimental nucleon masses measured in a lab.

DD2 is also constrained by several nuclear theory based studies.
Several studies have shown that nuclear energy bounds fall within the predictions of Chiral Effective Field theory.
Unlike the classical Lattimer-Swesty [cite] and the STOS [cite] EOS models that employ a single nucleus approximation, DD2 includes a detailed distributions of thousands of different nuclei.

DD2 also allows for the creation of stars whose masses are above the observations we have to-date.
Compared to observational constraints, DD2 can allow for masses as high as $\sim 2.4 M_\odot$, much greater than the observed $\sim 2 M_\odot$ neutron stars.
DD2, however, comes within $\sim 1 \textrm{km}$ of measured areal radii--being just slightly outside of with $13.22 \textrm{km}$, but within statistical uncertainties.

\section{FSU 2.1}

In addition to the relatively stiff equation of state DD2, FSU21 even stiffer with radii \textbf{TODO}



